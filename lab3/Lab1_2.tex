% Options for packages loaded elsewhere
\PassOptionsToPackage{unicode}{hyperref}
\PassOptionsToPackage{hyphens}{url}
%
\documentclass[
  12pt,
]{article}
\usepackage{amsmath,amssymb}
\usepackage{lmodern}
\usepackage{iftex}
\ifPDFTeX
  \usepackage[T1]{fontenc}
  \usepackage[utf8]{inputenc}
  \usepackage{textcomp} % provide euro and other symbols
\else % if luatex or xetex
  \usepackage{unicode-math}
  \defaultfontfeatures{Scale=MatchLowercase}
  \defaultfontfeatures[\rmfamily]{Ligatures=TeX,Scale=1}
\fi
% Use upquote if available, for straight quotes in verbatim environments
\IfFileExists{upquote.sty}{\usepackage{upquote}}{}
\IfFileExists{microtype.sty}{% use microtype if available
  \usepackage[]{microtype}
  \UseMicrotypeSet[protrusion]{basicmath} % disable protrusion for tt fonts
}{}
\makeatletter
\@ifundefined{KOMAClassName}{% if non-KOMA class
  \IfFileExists{parskip.sty}{%
    \usepackage{parskip}
  }{% else
    \setlength{\parindent}{0pt}
    \setlength{\parskip}{6pt plus 2pt minus 1pt}}
}{% if KOMA class
  \KOMAoptions{parskip=half}}
\makeatother
\usepackage{xcolor}
\usepackage[margin=1in]{geometry}
\usepackage{color}
\usepackage{fancyvrb}
\newcommand{\VerbBar}{|}
\newcommand{\VERB}{\Verb[commandchars=\\\{\}]}
\DefineVerbatimEnvironment{Highlighting}{Verbatim}{commandchars=\\\{\}}
% Add ',fontsize=\small' for more characters per line
\usepackage{framed}
\definecolor{shadecolor}{RGB}{248,248,248}
\newenvironment{Shaded}{\begin{snugshade}}{\end{snugshade}}
\newcommand{\AlertTok}[1]{\textcolor[rgb]{0.94,0.16,0.16}{#1}}
\newcommand{\AnnotationTok}[1]{\textcolor[rgb]{0.56,0.35,0.01}{\textbf{\textit{#1}}}}
\newcommand{\AttributeTok}[1]{\textcolor[rgb]{0.77,0.63,0.00}{#1}}
\newcommand{\BaseNTok}[1]{\textcolor[rgb]{0.00,0.00,0.81}{#1}}
\newcommand{\BuiltInTok}[1]{#1}
\newcommand{\CharTok}[1]{\textcolor[rgb]{0.31,0.60,0.02}{#1}}
\newcommand{\CommentTok}[1]{\textcolor[rgb]{0.56,0.35,0.01}{\textit{#1}}}
\newcommand{\CommentVarTok}[1]{\textcolor[rgb]{0.56,0.35,0.01}{\textbf{\textit{#1}}}}
\newcommand{\ConstantTok}[1]{\textcolor[rgb]{0.00,0.00,0.00}{#1}}
\newcommand{\ControlFlowTok}[1]{\textcolor[rgb]{0.13,0.29,0.53}{\textbf{#1}}}
\newcommand{\DataTypeTok}[1]{\textcolor[rgb]{0.13,0.29,0.53}{#1}}
\newcommand{\DecValTok}[1]{\textcolor[rgb]{0.00,0.00,0.81}{#1}}
\newcommand{\DocumentationTok}[1]{\textcolor[rgb]{0.56,0.35,0.01}{\textbf{\textit{#1}}}}
\newcommand{\ErrorTok}[1]{\textcolor[rgb]{0.64,0.00,0.00}{\textbf{#1}}}
\newcommand{\ExtensionTok}[1]{#1}
\newcommand{\FloatTok}[1]{\textcolor[rgb]{0.00,0.00,0.81}{#1}}
\newcommand{\FunctionTok}[1]{\textcolor[rgb]{0.00,0.00,0.00}{#1}}
\newcommand{\ImportTok}[1]{#1}
\newcommand{\InformationTok}[1]{\textcolor[rgb]{0.56,0.35,0.01}{\textbf{\textit{#1}}}}
\newcommand{\KeywordTok}[1]{\textcolor[rgb]{0.13,0.29,0.53}{\textbf{#1}}}
\newcommand{\NormalTok}[1]{#1}
\newcommand{\OperatorTok}[1]{\textcolor[rgb]{0.81,0.36,0.00}{\textbf{#1}}}
\newcommand{\OtherTok}[1]{\textcolor[rgb]{0.56,0.35,0.01}{#1}}
\newcommand{\PreprocessorTok}[1]{\textcolor[rgb]{0.56,0.35,0.01}{\textit{#1}}}
\newcommand{\RegionMarkerTok}[1]{#1}
\newcommand{\SpecialCharTok}[1]{\textcolor[rgb]{0.00,0.00,0.00}{#1}}
\newcommand{\SpecialStringTok}[1]{\textcolor[rgb]{0.31,0.60,0.02}{#1}}
\newcommand{\StringTok}[1]{\textcolor[rgb]{0.31,0.60,0.02}{#1}}
\newcommand{\VariableTok}[1]{\textcolor[rgb]{0.00,0.00,0.00}{#1}}
\newcommand{\VerbatimStringTok}[1]{\textcolor[rgb]{0.31,0.60,0.02}{#1}}
\newcommand{\WarningTok}[1]{\textcolor[rgb]{0.56,0.35,0.01}{\textbf{\textit{#1}}}}
\usepackage{graphicx}
\makeatletter
\def\maxwidth{\ifdim\Gin@nat@width>\linewidth\linewidth\else\Gin@nat@width\fi}
\def\maxheight{\ifdim\Gin@nat@height>\textheight\textheight\else\Gin@nat@height\fi}
\makeatother
% Scale images if necessary, so that they will not overflow the page
% margins by default, and it is still possible to overwrite the defaults
% using explicit options in \includegraphics[width, height, ...]{}
\setkeys{Gin}{width=\maxwidth,height=\maxheight,keepaspectratio}
% Set default figure placement to htbp
\makeatletter
\def\fps@figure{htbp}
\makeatother
\setlength{\emergencystretch}{3em} % prevent overfull lines
\providecommand{\tightlist}{%
  \setlength{\itemsep}{0pt}\setlength{\parskip}{0pt}}
\setcounter{secnumdepth}{-\maxdimen} % remove section numbering
\usepackage[OT4]{polski}
\usepackage[utf8]{inputenc}
\usepackage{graphicx}
\usepackage{float}
\ifLuaTeX
  \usepackage{selnolig}  % disable illegal ligatures
\fi
\IfFileExists{bookmark.sty}{\usepackage{bookmark}}{\usepackage{hyperref}}
\IfFileExists{xurl.sty}{\usepackage{xurl}}{} % add URL line breaks if available
\urlstyle{same} % disable monospaced font for URLs
\hypersetup{
  pdftitle={Raport z Laboratoriów 3},
  pdfauthor={Adrian Siwak, numer albumu 242084},
  hidelinks,
  pdfcreator={LaTeX via pandoc}}

\title{Raport z Laboratoriów 3}
\author{Adrian Siwak, numer albumu 242084}
\date{2023-03-22}

\begin{document}
\maketitle

{
\setcounter{tocdepth}{2}
\tableofcontents
}
W celu zbadania sprawności pewnej elektrowni wiatrowej wykonano 25
pomiarów prędkości wiatru (zmienna v) i odpowiadającego jej napięcia
prądu stałego, wytwarzanego przez tę elektrownię (zmienna DC).

Wczytaj do pakietu dane z pliku `elektrownia.xlsx', zawierającego te
pomiary.

\begin{Shaded}
\begin{Highlighting}[]
\FunctionTok{library}\NormalTok{(xtable)}
\FunctionTok{library}\NormalTok{(openxlsx)}
\NormalTok{dane}\OtherTok{\textless{}{-}}\FunctionTok{read.xlsx}\NormalTok{(}\StringTok{"elektrownia.xlsx"}\NormalTok{)}
\end{Highlighting}
\end{Shaded}

\hypertarget{section}{%
\section{1}\label{section}}

Wykonaj wykres rozproszenia dla próby (v1, DC1), . . . ,(v25, DC25) i
oblicz współczynnik korelacji próbkowej. Czy zależność między zmiennymi
v a DC ma charakter liniowy?

\begin{Shaded}
\begin{Highlighting}[]
\FunctionTok{plot}\NormalTok{(dane)}
\end{Highlighting}
\end{Shaded}

\begin{figure}[H]

{\centering \includegraphics{Lab1_2_files/figure-latex/zad1-1} 

}

\caption{\label{fig:wykres_rozrzutu_zad1}Wykres rozproszenia ZADANIE 1}\label{fig:zad1}
\end{figure}

\begin{Shaded}
\begin{Highlighting}[]
\NormalTok{corr}\OtherTok{=}\FunctionTok{cor}\NormalTok{(dane}\SpecialCharTok{$}\NormalTok{v,dane}\SpecialCharTok{$}\NormalTok{DC)}
\end{Highlighting}
\end{Shaded}

Wrtość współczynnika korelacji wynosi 0.935143430666912, więc wnioskować
można o silnej korelacji,

wykres nie ma jednak charakteru liniowego.

\hypertarget{section-1}{%
\section{2}\label{section-1}}

W modelu regresji liniowej
\(DC = \beta_{0} + \beta_{1} \cdot v + \mathcal{E}\) opisującym
zależność między zmienną objaśnianą \(DC\) i zmienną objaśniającą \(v\),
wyznacz estymatory najmniejszych kwadratów \(\hat{\beta_{0}}\) i
\(\hat{\beta_{1}}\) parametrów \(\beta_{0}\) i \(\beta_{1}\).

\begin{Shaded}
\begin{Highlighting}[]
\NormalTok{DC}\OtherTok{\textless{}{-}}\NormalTok{dane}\SpecialCharTok{$}\NormalTok{DC}
\NormalTok{v}\OtherTok{\textless{}{-}}\NormalTok{dane}\SpecialCharTok{$}\NormalTok{v}
\NormalTok{model}\OtherTok{\textless{}{-}}\FunctionTok{lm}\NormalTok{(DC}\SpecialCharTok{\textasciitilde{}}\NormalTok{v)}
\NormalTok{b\_0}\OtherTok{=}\NormalTok{model}\SpecialCharTok{$}\NormalTok{coefficients[}\DecValTok{1}\NormalTok{]}
\NormalTok{b\_1}\OtherTok{=}\NormalTok{model}\SpecialCharTok{$}\NormalTok{coefficients[}\DecValTok{2}\NormalTok{]}
\end{Highlighting}
\end{Shaded}

Estymatory najmniejszych kwadratów \(\hat{\beta_{0}}\) i
\(\hat{\beta_{1}}\) mają wartości: \(\hat{\beta_{0}}\) =
0.130875130898235 \(\hat{\beta_{1}}\) = 0.24114886971653

\hypertarget{section-2}{%
\section{3}\label{section-2}}

Wyznacz \(R^{2}\), \(\hat{\sigma^{2}}\) oraz p-wartość dla testu
\(H_{0} : \beta_{1} = 0\) vs \(H_{1} : \beta_{1} \neq 0\); Czy na
poziomie istotaności \(\alpha = 0,05\) należy odrzucić \(H_{0}\)?

\begin{Shaded}
\begin{Highlighting}[]
\NormalTok{summary}\OtherTok{=}\FunctionTok{summary}\NormalTok{(model)}

\NormalTok{sigma\_squared\_hat}\OtherTok{=}\NormalTok{(summary}\SpecialCharTok{$}\NormalTok{sigma)}\SpecialCharTok{**}\DecValTok{2}

\NormalTok{R\_2}\OtherTok{=}\NormalTok{summary}\SpecialCharTok{$}\NormalTok{r.squared}

\NormalTok{T\_stat}\OtherTok{=}\FunctionTok{summary}\NormalTok{(model)}\SpecialCharTok{$}\NormalTok{coefficients[}\DecValTok{2}\NormalTok{,}\DecValTok{3}\NormalTok{]}
\NormalTok{t}\OtherTok{=}\FunctionTok{qt}\NormalTok{(}\FloatTok{0.975}\NormalTok{,}\DecValTok{23}\NormalTok{)}
\NormalTok{p\_value}\OtherTok{=}\FunctionTok{summary}\NormalTok{(model)}\SpecialCharTok{$}\NormalTok{coefficients[}\DecValTok{2}\NormalTok{,}\DecValTok{4}\NormalTok{]}
\end{Highlighting}
\end{Shaded}

Wartość statystyki \(T\) wynosi 12.6592675639296. Hipoteza zerowa może
być odrzucona na poziomie istotności \(\alpha = 0,05\) gdy wartość
bezwzględna statystyki T jest większa niż \(t_{\alpha/2,n-2}\) - kwantyl
rzędu \(1-\alpha/2\) rozkładu t-Studenta z n-2 stopniami swobody. W tym
przypadku: \(n=25\), \(n-2=23\), \(1-\alpha/2 = 0,975\),
\(t_{\alpha/2,n-2}\) = 2.06865761041905. 12.6592675639296 \textgreater{}
2.06865761041905 więc hipotezę zerową możemy odrzucić na poziomie
istotności \(\alpha = 0,05\). Ten test ma p-wartość równą
7.54552540506968e-12. Wartość \(R^{2}\) wynosi 0.874493235919481,
natomiast \(\hat{\sigma^{2}}\) wynosi 0.0557205723172969.

\hypertarget{section-3}{%
\section{4}\label{section-3}}

Po ponownym przeanalizowaniu wykresu rozproszenia dla zmiennych \(DC\) i
\(v\) stwórz nową zmienną objaśniającą \(\hat{v}\), będącą jakąś funkcją
zmiennej \(v\). Zmienną \(\hat{v}\) dobierz tak, by wykres rozproszenia
dla zmiennych \(DC\) i \(\hat{v}\) był bardziej zbliżony do wykresu
liniowego niż wykres rozproszenia dla zmiennych \(DC\) i \(v\).

\begin{Shaded}
\begin{Highlighting}[]
\NormalTok{v\_new}\OtherTok{\textless{}{-}}\NormalTok{v}\SpecialCharTok{**}\NormalTok{(.}\DecValTok{1}\NormalTok{)}
\FunctionTok{plot}\NormalTok{(v\_new,DC)}
\end{Highlighting}
\end{Shaded}

\begin{figure}[H]

{\centering \includegraphics{Lab1_2_files/figure-latex/zadanie_4-1} 

}

\caption{\label{fig:wykres_rozrzutu_zad4}Wykres rozproszenia ZADANIE 4}\label{fig:zadanie_4}
\end{figure}

\hypertarget{section-4}{%
\section{5}\label{section-4}}

W modelu regresji liniowej
\(DC = \beta_{0} + \beta_{1} \cdot v + \mathcal{E}\) wyznacz estymatory
najmniejszych kwadratów \(\hat{\beta_{0}}\) i \(\hat{\beta_{1}}\)
parametrów \(\beta_{0}\) i \(\beta_{1}\). Jeśli współczynnik
determinacji \(R^{2}\) w „nowym'' modelu jest większy od \(R^{2}\) w
„starym'' modelu, to przejdź do następnego punktu. W przeciwnym razie
wróć do poprzedniego punktu.

\begin{Shaded}
\begin{Highlighting}[]
\NormalTok{model2}\OtherTok{\textless{}{-}}\FunctionTok{lm}\NormalTok{(DC}\SpecialCharTok{\textasciitilde{}}\NormalTok{v\_new)}
\NormalTok{b\_0\_2}\OtherTok{=}\NormalTok{model}\SpecialCharTok{$}\NormalTok{coefficients[}\DecValTok{1}\NormalTok{]}
\NormalTok{b\_1\_2}\OtherTok{=}\NormalTok{model}\SpecialCharTok{$}\NormalTok{coefficients[}\DecValTok{2}\NormalTok{]}

\NormalTok{summary2}\OtherTok{=}\FunctionTok{summary}\NormalTok{(model2)}

\NormalTok{R\_2\_2}\OtherTok{=}\NormalTok{summary2}\SpecialCharTok{$}\NormalTok{r.squared}
\end{Highlighting}
\end{Shaded}

W nowym modelu wartość

0.951441909174668 \textgreater{} 0.874493235919481

\$\hat{\beta_{0}} =\$0.130875130898235

\$\hat{\beta_{1}} =\$0.24114886971653

\hypertarget{section-5}{%
\section{6}\label{section-5}}

Wskaż lepszy z modeli i uzasadnij swój wybór.

Ponieważ wartość \(R^{2}\) jest większa w ``nowym'' modelu, a wykres
rozrzutu jest bardziej liniowy, jest on modelem lepszym.

\hypertarget{section-6}{%
\section{7}\label{section-6}}

Porównaj obserwowane i prognozowane przez oba modele wartości zmiennej
DC.

\begin{Shaded}
\begin{Highlighting}[]
\NormalTok{Nr\_obserwacji}\OtherTok{\textless{}{-}}\DecValTok{1}\SpecialCharTok{:}\DecValTok{25}
\NormalTok{obserwowane\_DC}\OtherTok{\textless{}{-}}\NormalTok{DC}
\NormalTok{DC\_w\_starym\_modelu}\OtherTok{\textless{}{-}}\FunctionTok{predict}\NormalTok{(model)}
\NormalTok{DC\_w\_nowym\_modelu}\OtherTok{\textless{}{-}}\FunctionTok{predict}\NormalTok{(model2)}
\NormalTok{tabela}\OtherTok{\textless{}{-}}\FunctionTok{data.frame}\NormalTok{(Nr\_obserwacji,obserwowane\_DC,DC\_w\_starym\_modelu,DC\_w\_nowym\_modelu)}
\FunctionTok{print}\NormalTok{(}\FunctionTok{xtable}\NormalTok{(tabela), }\AttributeTok{type =} \StringTok{"latex"}\NormalTok{, }\AttributeTok{table.placement =} \StringTok{"H"}\NormalTok{, }\AttributeTok{comment=}\ConstantTok{FALSE}\NormalTok{)}
\end{Highlighting}
\end{Shaded}

\begin{verbatim}
## \begin{table}[H]
## \centering
## \begin{tabular}{rrrrr}
##   \hline
##  & Nr\_obserwacji & obserwowane\_DC & DC\_w\_starym\_modelu & DC\_w\_nowym\_modelu \\ 
##   \hline
## 1 &   1 & 1.58 & 1.34 & 1.44 \\ 
##   2 &   2 & 1.82 & 1.58 & 1.70 \\ 
##   3 &   3 & 1.06 & 0.95 & 0.90 \\ 
##   4 &   4 & 0.50 & 0.78 & 0.60 \\ 
##   5 &   5 & 2.24 & 2.54 & 2.45 \\ 
##   6 &   6 & 2.39 & 2.47 & 2.40 \\ 
##   7 &   7 & 2.29 & 2.43 & 2.38 \\ 
##   8 &   8 & 0.56 & 0.87 & 0.76 \\ 
##   9 &   9 & 2.17 & 2.10 & 2.14 \\ 
##   10 &  10 & 1.87 & 1.63 & 1.74 \\ 
##   11 &  11 & 0.65 & 0.83 & 0.69 \\ 
##   12 &  12 & 1.93 & 1.66 & 1.78 \\ 
##   13 &  13 & 1.56 & 1.24 & 1.32 \\ 
##   14 &  14 & 1.74 & 1.53 & 1.65 \\ 
##   15 &  15 & 2.09 & 1.92 & 2.00 \\ 
##   16 &  16 & 1.14 & 1.00 & 0.98 \\ 
##   17 &  17 & 2.18 & 2.02 & 2.08 \\ 
##   18 &  18 & 2.11 & 2.25 & 2.25 \\ 
##   19 &  19 & 1.80 & 1.82 & 1.92 \\ 
##   20 &  20 & 1.50 & 1.45 & 1.56 \\ 
##   21 &  21 & 2.30 & 2.33 & 2.30 \\ 
##   22 &  22 & 2.31 & 2.59 & 2.47 \\ 
##   23 &  23 & 1.19 & 1.12 & 1.16 \\ 
##   24 &  24 & 1.14 & 1.08 & 1.11 \\ 
##   25 &  25 & 0.12 & 0.72 & 0.47 \\ 
##    \hline
## \end{tabular}
## \end{table}
\end{verbatim}

\end{document}
